\documentclass[12pt]{article}

\usepackage[english]{babel}
\usepackage{indentfirst}
\usepackage{tabularx}

% - Margin - 1 inch on all sides
\usepackage[letterpaper]{geometry}
\usepackage[utf8]{inputenc}
\geometry{top=1.0in, bottom=1.0in, left=1.0in, right=1.0in}



% FancyHDR settings
\setlength{\headheight}{15.2pt}
\usepackage{fancyhdr}
\pagestyle{fancy}
\fancyhf{}

\fancyhead[L]{James Harbour}
\fancyhead[C]{\emph{Revolution is not a Dinner Party}: Dialectic Journal}

\newcommand{\vc}[4]{
  \item Ch #1.#2
  \begin{itemize}
    \item #3
    \item #4
  \end{itemize}
}


% Begin Document
\begin{document}




\begin{tabularx}{\textwidth}{ |X|X| }
  \hline
  ``How could anyone be more dear than my father? Would Chairman Mao let me put ponytails on him?'' (10) & This quotation provides a stark example of Ling's naiveté and thus creates the environment for a Bildungsroman, i.e. a coming-of-age story. By showcasing Ling's lack of knowledge about political ideologies, the story prepares for political strife as a main conflict.  \\
  \hline
  ``We used to recieve letters from overseas, too, but . . . after receiving a letter that had been opened, Mother became nervous and told father to stop writing to his friends. Why would anyone want to open our letters?'' (24) & Similar to the previous quotation, this passage displays Ling's lack of knowledge about established idealogical control; however, it does display a factor of curiosity in her thinking. By implanting this interest in the reasoning behind the status quo, the author could create a narrative in which this curiosity leads to conflict.  \\
  \hline
  ```Hopefully the mangoes will keep him busy for the night.' . . . I thought my parents did not like Comrade Li because he bought things from us. I was wrong'' (35). & This quotation finally introduces the antagonist of the story whilst hinting towards some kind of plan being set in motion. \\
  \hline
  ```They`re arresting an undercover enemy,' I said. My heart pounded. `What undercver enemy? Who is it?''' (50) & After introducting the main antagonist in the last chapter, the author finally implements a moment of hightened stakes and thus allows the plot to move forward in a more succinct manner than before. \\
  \hline
  ``To celebrate the victory of the Communist Revolution, many of the streets had been renamed, such as Big Liberation Road, Victory Road, Workers and Parents Road, and Red Five Stars Road'' (56). & It astonishes me that the propoganda of the Communist party permeated even the street names. The contiual use of them prompts a differentiation between the actions of the party and the party themselves. \\
  \hline
  ``That night, I had a horrible dream. Father was taken away by a mob without faces'' (68). & This premonition likely foreshadows a dramatic shift in the narrative when Ling's parents are taken away for being ``enemies of the state.'' \\
  \hline
  ``The last time we had gone to Hing Shing, someone had sealed off the doors with long strips of red paper that read \emph{BOURGEOIS NEST}'' (81). & This situation appears similar to pre-WWII Nazi Germany's discrimination against Jewish business owners. It also evokes memories of Solzhenitsyn's \emph{The Gulag Archipelago}. \\
  \hline
  ``The flames leaped out as if trying to grab us. Comrade Li pulled Father's books from the shelves and threw them into the fire'' (95). & And so the conflict begins. The symbol of flames gripping Ling and her family represents the fire of the revolutionaries metaphorically burning her family. \\
  \hline
  ```Magic won't help.' Niu banged his fist on the table. `The only way out is to escape!''' (114) & This declaration provides a clear direction for the story, and I expect there to be a large amount of tragedy in following this path. \\
  \hline
  ``I'd never met Chairman Mao. I doubted he would take care of me when I was sick or sing Engish songs with me. He could never be dearer than my parents'' (124). & Having begun to realize the reality of her situation, Ling makes this stark declaration of resistence to the dictatorship of the proletariat. This mindset might set up the story for Ling to escape from China, however this is still unclear at the moment. \\
  \hline
  ``'' () & - \\
  \hline
  ``'' () & - \\
  \hline


\end{tabularx}
\vspace{2em}
\center{Vocab}
\begin{itemize}

  \vc{1}{1}{There are no vocabulary in this chapter with which I am unfamiliar}{}

  \vc{1}{2}{The Communist Party (I already know the definition, but for the sake of a grade lets assume I don't): Political party which guides the political education and development of the working class (proletariat).Tthe communist party exercises power through the dictatorship of the proletariat.}{There are no other words in this chapter which any decently educated seventh grader would not know.}

  \vc{1}{3}{Mung Bean: a plant species in the legume family. The mung bean is mainly cultivated in East Asia, Southeast Asia and Indian subcontinent. It is used as an ingredient in both savory and sweet dishes}{fen (I assume that it is a unit of currency; however, I want points): a unit of currency used in Greater China, including People's Republic of China, Republic of China (Taiwan), Hong Kong and Macao.}

  \vc{1}{4}{Han Bridge: a cable-stayed swing bridge in Da Nang, Vietnam, on the west side of the Hàn River.}{Han River: The Guangdong river.}

  \vc{1}{5}{revolutionary operas: a series of shows planned and engineered during the Cultural Revolution by Jiang Qing, the wife of Chairman Mao Zedong. They were considered revolutionary and modern in terms of thematic and musical features when compared with traditional Chinese operas.}{Cultural Revolution: a sociopolitical movement in the People's Republic of China that was launched by Mao Zedong. Its stated goal was to preserve Chinese Communism by purging remnants of capitalist and traditional elements from Chinese society, and to re-impose Mao Zedong Thought as the dominant ideology in the Party.}

  \vc{1}{6}{burgeouis: of or characteristic of the middle class, typically with reference to its perceived materialistic values or conventional attitudes.}{There are no other unfamiliar vocabulary in this chapter.}

  \vc{1}{7}{Red Guard: a mass student-led paramilitary social movement mobilized and guided by Mao Zedong during the first phase of the Chinese Cultural Revolution.}{There are no other unfamiliar vocabulary in this chapter (I do not wish to dodge work; however, I also do not wish to paint myself as an idiot who does not understand the basic vocabulary of my native language, English).}

  \vc{1}{8}{Again, this chapter's vocabulary does not provide any challenge; I apologize for not being able to create an entry, please understand.}{}

  \vc{2}{1}{Canton: A city in southern China.}{``the dial (108)'': I assume this is referring to a radio.}

  \vc{2}{2}{People's Liberation Army: the armed forces of the People's Republic of China and its founding and ruling political party, the Communist Party of China}{Young Pioneers: a mass youth organization for children aged six to fourteen in the People's Republic of China. The Young Pioneers of China is run by the Communist Youth League, an organization of older youth that comes under the Communist Party of China}
\end{itemize}




\end{document}
