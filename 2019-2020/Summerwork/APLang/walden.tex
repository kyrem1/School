\documentclass[12pt]{article}

%%%%%%%%%%%%%%%%%%%%%%%%%%%%%%%%%%-- Settings --%%%%%%%%%%%%%%%%%%%%%%%%%%%%%%%%%%%%%%%%%%%
\usepackage[english]{babel}

% - Margin - 1 inch on all sides
\usepackage[letterpaper]{geometry}
\usepackage[utf8]{inputenc}
\geometry{top=1.0in, bottom=1.0in, left=1.0in, right=1.0in}

% - Double Spacing -
\usepackage{setspace}
\doublespacing
\setstretch{2.00}

% FancyHDR settings
\setlength{\headheight}{15.2pt}
\usepackage{fancyhdr}
\pagestyle{fancy}
\fancyhf{}

% Indent First paragraph
\usepackage{indentfirst}

% Title Settings
\title{
    \vspace{2in}
    \textmd{\textbf{AP Language and Composition:\ Summerwork}}\\
    \normalsize\vspace{0.1in}\small{Due\ on\ August 12, 2019\ at 8:00pm}\\
    \vspace{3in}
}
\author{James Harbour}

% \tq{TQ num}{CH num}
\newcommand{\tq}[2]{
    \fancyhead[L]{\emph{Walden}: Reading Journals}
    \fancyhead[R]{TQ:\ #1 $\vert$ CH: #2}
}

\renewcommand{\footnoterule}{%
  \kern -3pt
  \hrule width \textwidth height 0.5pt
  \kern 2pt
}

%%%%%%%%%%%%%%%%%%%%%%%%%%%%%%%%%%-- Assignment --%%%%%%%%%%%%%%%%%%%%%%%%%%%%%%%%%%%%%%%%%%%

% http://www.gutenberg.org/files/205/205-h/205-h.htm

% 1. Describe the persona of the narrator (though Thoreau uses first person narration, he creates a narrative persona): what is revealed through tone, diction, syntax, etc. How does he create credibility (ethos: phronesis, arete and eunoia) and develop a relationship with the audience?
% 2. Analyze the relationship between exigence and kairos: how does Thoreau take advantage of the given situation during this time period and respond using the best available means of persuasion?
% 3. Who is the audience for the text (what are their beliefs, biases, background knowledge, etc.), and how does Thoreau adapt his message to this audience?
% 4. What rhetorical devices and strategies do you notice (refer to the rhetorical terms which follow), and how do these contribute to Thoreau’s argument and purpose?
% 5. How do aspects of narrative development contribute to Thoreau’s argument and purpose?
% 6. Analyze the text’s argument: Is it implicit or explicit? How does Thoreau back his claim?
% 7. Analyze Thoreau’s purpose: is it a question of fact, definition, quality or policy (stasis theory)? How do the rhetorical choices Thoreau makes relate to his purpose?
% 8. Conduct a detailed analysis of a short passage in terms of syntax. Describe the syntax and relate to Thoreau’s purpose.
% 9. Analyze a time when your reading changed (for example, you recognize a pattern, the text suddenly seems to be about something different from what you originally thought, you discover that you were misreading, you realize Thoreau has introduced a new context or perspective, or you were surprised or puzzled). Describe your thought process and realizations.
% 10. Analyze Thoreau’s philosophy of transcendentalism using textual evidence.
% 11. Analyze details that seem important and that make you take a second look: how does this relate to Thoreau’s purpose?
% 12. Analyze how Thoreau uses ambiguity and relate to his purpose.
% 13. Select a passage in which either dialogue (see the chapter “Brute Neighbors”) or quotations are used for a particular effect and analyze the effect.
% 14. Provide examples of an effective aspect of Thoreau’s style from two different passages in the text and analyze why it is effective, comparing and contrasting its use in the two passages.
% 15. Note your impressions of the way the text ended, and analyze how this affects your reading of the text.


%%%%%%%%%%%%%%%%%%%%%%%%%%%%%%%%%%-- Document --%%%%%%%%%%%%%%%%%%%%%%%%%%%%%%%%%%%%%%%%%%%

% Begin Document
\begin{document}
\maketitle
\pagebreak
\raggedright\setlength{\parindent}{0.5in}


\tq{13}{1 - Economy}
% 13. Select a passage in which either dialogue (see the chapter “Brute Neighbors”) or quotations are used for a particular effect and analyze the effect.

After listing the various ways in which mankind suffers from uncertainty, Thoreau follows with a quotation from Confucius providing a mindset able to transcend this anxiety and affect positive change. Directly following a celebration of the miracle of contemplation, Thoreau writes, ``Confucius said, `To know that we know what we know, and that we do not know what we do not know, that is true knowledge'" (7). Thoreau utilizes the ethos of Confucius to emphasize the quintessential example of an individual with the ability to surpass the strain of everyday life. This ideal alleviates the pessimism of the negative musings in the preceding paragraph through the provision of a concrete mindset to strive towards. This paragraph, taken as a whole, formulates a method of resisting descent into nihilismtic pessimism whilst simultaneously building its ethos. Thoreau first presents his understanding of the reasons behind why mankind ``commit[s] [it]sel[f] to uncertainties'' (7) to later strengthen the credibility of his solution: that the act of contemplation directly combats nihilistic pessimism.

\newpage
\tq{11}{2 - Where I Lived, and What I Lived For}

% 11. Analyze details that seem important and that make you take a second look: how does this relate to Thoreau’s purpose?

For an enlightened, logical figure such as Thoreau, the avoidance of contradictory arguments should stand at the forefront of his mind when writing. Thus, Thoreau's avoidance of the neccessity of counting provides a peculiar example of purposefully contradicting arguments/statements. In his chapter describing the general setting of the rest of \emph{Walden}, Thoreau declares, ``Our life is frittered away by detail. An honest man has hardly need to count more than his ten fingers, or in extreme cases he may add his ten toes, and lump the rest. Simplicity, simplicity, simplicity!'' (51) Before reaching the contradicting argument to this quotation, it appears rather vague and unintersting with the exception of Thoreau's inclusion of a rhetorical fragment. Nonetheless, the knowledge of Thoreau's previous employment as a surveyor would immediately raise red flags, as was likely the intention. The contradiction to this quotation appears in his chapter on Walden Pond in the winter when Thoreau precisely measure the pond's dimensions: ``In one instance, on a line arbitrarily chosen, the depth did not vary more than one foot in thirty rods; and generally, near the middle, I could calculate the variation for each one hundred feet in any direction beforehand within three or four inches. . . . I had mapped the pond by the scale of ten rods to an inch, and put down the soundings, more than a hundred in all'' (156). Thoreau's use of precise measurements clearly stands in opposition to his avoidance of counting. By including this contradiction, Thoreau suggests a lack of rigidity in his philosophy and seeks to relay the flexibility in the mode of life he presents.

\newpage
\tq{1}{3 - Reading}

% 1. Describe the persona of the narrator (though Thoreau uses first person narration, he creates a narrative persona): what is revealed through tone, diction, syntax, etc. How does he create credibility (ethos: phronesis, arete and eunoia) and develop a relationship with the audience?

Through the use of first person narration, Thoreau crafts the narrative persona of an elder imparting knowledge to younger generations through both a description of his own life experiences and the creation of a feeling of collective responsibility with the use of the first person plural pronoun. An example of this dualistic impartment and acceptance of responsibility rises during Thoreau's admittance of his seldom reading of classical authors such as Plato: ``His Dialogues, which contain what was immortal in him, lie on the next shelf, and yet I never read them. . . . We should be as good as worthies of antiquity, but partly by first knowing how good they were. We are a race of titmen, and soar but little higher in our intellectual flights than the columns of the daily paper'' (60). By creating a feeling of responsibility alongside the rest of mankind, Thoreau morphs this otherwise negative statement into an insight suggesting hope for the future. In this way, he creates ethos through eunoia; however, Thoreau also relies on other methods of forging his credibility. Often when Thoreau crafts an argument, he proceeds to explain the counterargument in detail showcasing phronesis as he posesses the experience to understand not only his argument, but also the opposite of his argument.

\newpage
\tq{8}{4 - Sounds}

% 8. Conduct a detailed analysis of a short passage in terms of syntax. Describe the syntax and relate to Thoreau’s purpose.3

Out of the many passages in \emph{Walden} lending themselves to syntactical analysis, Thoreau's most poigniant thus far appears near the beginning of his chapter on sounds: ``I had this advantage, at least, in my mode of life, over those who were obliged to look abroad for amusement, to society and the theatre, that my life itself was become my amusement and never ceased to be novel. It was a drama of many scenes and without an end'' (63). Thoreau's use of irregular syntax in this passage cements the stark contrast between the perpetual novelty of his everyday life and the fleeting amusement found by a society that seeks entertainment from the external. The first example of this peculiar prose arises in the syntactical choice to use ``was become'' instead of the present perfect ``has become.'' In doing so, Thoreau draws attention to the anteriority of his life becoming an amusement, thus increasing the magnitude of the period of novelty he finds throughout his time at Walden Pond and comparatively morphing the effect of external entertainment into that of infinitesmal significance. One other notable syntactial choice appears in Thoreau's literal separation of the theatre from society. Although a product of society, theatre represents society's attempt to escape the tedium of societal obligation; moreover, Thoreau sees this attempt as having failed due to its brevity and uses the subject as a metaphor to liken his life in nature to that of a drama which succeeds in creating perpetual entertainment.

\newpage
\tq{10}{5 \& 6 - Solitude and Visitors}

% 10. Analyze Thoreau’s philosophy of transcendentalism using textual evidence.

The primary argument against the philosophy of Transcendentalism suggests that the solitude of isolation degrades mental health whilst cultivating feelings of loneliness. Throreau implicitly counters this argument during his chapter on solitude: ``We are for the most part more lonely when we go abroad among men than when we stay in our chambers'' (75). At first glance, this contention appears flimsy at best; however, the argument's genuity appears when applied to life experience. Thoreau continues his rebuttal stating that ``[s]olitude is not measured by the miles of space that intervene between a man and his fellows'' (75). This statement of the conjugate of solitude combined with the previous comparison of loneliness in different types of isolation provides the ideal example of Thoreau's philosophy of Transcendentalism. The emphasis on transcendental knowledge appears in his definition's implicit proposition: that Thoreau discovered this meaning for himself. His remark on loneliness clearly displays Thoreau's belief that he is at his best when truly independent of society.

\newpage
\tq{6}{7 \& 8 - The Bean-Field and The Village}

% 6. Analyze the text’s argument: Is it implicit or explicit? How does Thoreau back his claim?

% TODO Grammar AF

Throughout \emph{Walden}, Thoreau mainly crafts implicit arguments due to his belief in the superiority of transcendental knowledge. He seeks to provide experiences tied to various conclusions; however, Thoreau purposefully omits the link between these two in an effort to force his readers to infer the effects of his experiences. Focusing in on the chapters relating to his bean-field and trips to the village, Thoreau provides multiple implicit arguments in which he indirectly supports the conclusion with his experiences. An example of one of these contentions arises after Thoreau lists the profits of his bean-field and concerns the clash between mankind's focus on its own labor versus the wellbeing of its posterity: ``Why concern ourselves so much about our beans for seed, and not be concerned at all about a new generation of men? We should really be fed and cheered if when we met a man we were sure to see that some of the qualities which I have named, . . . had taken root and grown in him'' (91). This statement displays Thoreau's realization of humanity's misplaced concern for its own wellbeing outweighing concern for its posterity. Despite not providing any immediate explanation for this conclusion, Thoreau places his support for this contention in the lessons resulting from the completion his bean farm harvest after which he declares, ``I will not plant beans and corn with so much industry another summer, but such seeds, . . . even with less toil and manurance, and sustain me'' (90). Through this declaration, Thoreau urges readers to infer his newfound desire to plant seeds for sustenance for the sake of both having more time to expand his knowledge and be able to provide more for visitors and those in the village alike. He places his conclusion two paragraphs later to further allow this idea foster and allow the reader to more easily realize the connection upon reaching Thoreau's conclusion.

\newpage
\tq{9}{9 - The Ponds}

% 9. Analyze a time when your reading changed (for example, you recognize a pattern, the text suddenly seems to be about something different from what you originally thought, you discover that you were misreading, you realize Thoreau has introduced a new context or perspective, or you were surprised or puzzled). Describe your thought process and realizations.

Throughout my reading of \emph{Walden}, I became accustomed to Thoreau's rhetoric style of providing his experiences then presenting his conclusions derived indirectly from said experiences. However, whilst reading Thoreau's chapter on the various ponds surrounding his residence, I noticed that this rhetorical style abrubtly disappeared. In its place resides simply imagery coupled with Thoreau's explanation of his fondness for Walden Pond specifically. At first, I believed that he would leave his argument at the tail-end of the chapter for the sake of magnitude; however, I was mistaken as Thoreau completely omits any argumentation from this chapter. The entire chapter contains a wealth of vivid imagery which appears to flow in a manner akin to a river. The closest substance to an argument is a philosophical description of the water of Walden Pond in which Thoreau writes, ``Yet a single glass of its water held up to the light is as colorless as an equal quantity of air. It is well known that a large plate of glass will habe a green tint, owing, as the makers say, to its `body,' but a small piece of the same will be colorless'' (98). Perhaps, Thoreau includes this entire chapter of imagery as a metaphorical gate to the more philisophical upcoming chapters.

\newpage
\tq{7}{10 \& 11 - Baker Farm and Higher Laws}

% 7. Analyze Thoreau’s purpose: is it a question of fact, definition, quality or policy (stasis theory)? How do the rhetorical choices Thoreau makes relate to his purpose?

Through \emph{Walden}, Thoreau seeks to provide an answer to which mode of being provides the most satisfaction in life. Under Stasis theory, this question is obviously one of policy which suggests actions to take for the betterment of life. During his chapter on higher laws, Thoreau crafts an argument giving a clear indication of his purpose: ``If the day and the night are such that you greet them with joy, and life emits a fragrance like flowers and sweet-scented herbs, is more elastic, more starry, more immortal,—that is your success. All nature is your congratulation, and you have cause momentarily to bless yourself'' (118). Through this argument, Thoreau provides a goal with which to orient towards and declares the merit of doing so. In regards to his rhetorical choices, Thoreau's providing conclusions with hidden links to his experiences lends itself to the answering of questions of policy by forcing the reader to further ponder his argument and thus increasing the chance of a change in future actions. An example of this effect arises during his's encounter with John Field in which Thoreau writes, ``here you could get tea, and coffee, and meat every day. But the only true America is that country where you are at liberty to pursue such a mode of life as may enable you to do without these'' (112). By placing the suffering of John Field before this quotation, Thoreau urges the reader to infer that Field's suffering derives from his dependence upon the village's luxuries. Upon reaching this declaration, Thoreau's priming takes effect by linking a now present element of the reader's thought process to a Transcendentalist interpretation of America.

\newpage
\tq{3}{12 - Brute Neighbors}

% 3. Who is the audience for the text (what are their beliefs, biases, background knowledge, etc.), and how does Thoreau adapt his message to this audience?

Thoreau himself states that he wrote \emph{Walden} to answer the wealth of inquiries he would recieve concerning his time at Walden Pond after his lectures. However, Thoreau actually targets an even more specific audience: young men seeking a higher mode of living. Thoreau attempts to further pinpoint those for whom this desire has spawned on account of a newfound dissatisfaction in their current modus operandi. One clear intimation towards the nature of his audience appears in Thorea's sarcastic comparison of an ant battle to the Revolutionary War: ``I have no doubt that it was a principle they fought for, as much as our ancestors, and not to avoid a three–penny tax on their tea; and the results of this battle will be as important and memorable to those whom it concerns as those of the battle of Bunker Hill, at least'' (125). This passive aggression following Thoreau's vivid description of the ant battle emphasizes his distaste for the Revolutionary War whilst simultaneously allowing his audience to follow the train of thought leading to this deep-seated abhorrence. However, an understanding of the quintessence of this passage only manifests in readers whom already possess a predisposition towards agreeing with anti-war sentiments.

\newpage
\tq{4}{13 - House Warming}

% 4. What rhetorical devices and strategies do you notice (refer to the rhetorical terms which follow), and how do these contribute to Thoreau’s argument and purpose?

Thoreau often utilizes the varying situational effects of asyndetons to minimize distractions from his purpose of presenting a higher mode of living whilst still satisfying inquiries about his stay at Walden Pond. The use case of asyndeton increasing a description's magnitude appears during Thoreau's portrayal of his dream house stating that it ``shall still consist of only one room, a vast, rude, substantial, primitive hall, without ceiling or plastering, with bare rafters and purlins supporting a sort of lower heaven over one's head-useful to keep off rain and snow, where the king and queen posts stand out to receive your homage, when you have done reverence to the prostrate Saturn of an older dynasty on stepping over the sill'' (132). In this depiction, Thoreau uses asyndeton to emphasize the grandiosity of the house without sacrificing eloquence. Through this emphasis, Thoreau prevents any misconceptions that his following critism of hospitality derives from luxury, and thus he allows the reader to ponder the more fruitful implications of his critique. Thoreau's other major use of asyndeton involves the hastening of details inculuded solely for the sake of providing a complete account of his two years at Walden Pond. An example of this use case arises while Thoreau prepares for winter via plaster ``I remembered the story of a conceited fellow, who, . . . Venturing one day to substitute deeds for words, he turned up his cuffs, seized a plasterer's board, . . . made a bold gesture thitherward; and straightway, to his complete discomfiture, received the whole contents in his ruffled bosom. . . . I learned the various casualties to which the plasterer is liable'' (133). Although Thoreau dedicates almost an entire paragraph to the dangers of plastering, his clever use of asyndeton minimizes any impact to his overall purpose by speeding up the process of wading through the story.

\newpage
\tq{5}{14 - Former Inhabitants and Winter Visitors}

% 5. How do aspects of narrative development contribute to Thoreau’s argument and purpose?

Each chapter, whilst crafting arguments relating to his purpose of presenting a higher mode of living, Thoreau also includes narrative development in his surroundings to build his retelling of his time in Walden Pond. He uses these details to either provide a refreshing break from arguement or indirectly support an argument. An example of this use of narrative elements to provide a recess from argumentation arises when Thoreau mentions the arrival of winter: ``I weathered some merry snow-storms, and spent some cheerful winter evenings by my fireside, while the snow whirled wildly without. . . . The elements, however, abetted me in making a path through the deepest snow in the woods, for when I had once gone through the wind blew the oak leaves into my tracks, where they lodged, and by absorbing the rays of the sun melted the snow, and so not only made a my bed for my feet, but in the night their dark line was my guide'' (139). Not only does this detail progress the climate, but also introduces the route in which Thoreau travels to the town. With this introduction, Thoreau primes the reader for a shift in subject matter to that of people he meets on the road to the town.

\newpage
\tq{12}{15 - Winter Animals}

% 12. Analyze how Thoreau uses ambiguity and relate to his purpose.

Thoreau's specific use of ambiguity to add a feeling of wonder to his descriptions of nature increases the genuity of his argument about an ideal mode of living. The inherent fascination behind his descriptions of everyday natural occurences showcases the effects of his stay at Walden Pond combined with his Transcendentalist ideals. This mystical feeling of awe often appears in  description of animals such as when Thoreau recounts a dialogue between a goose and cat-owl: ``It was one of the most thrilling discords I ever heard. And yet, if you had a discriminating ear, there were in it the elements of a concord such as these plains never saw nor heard'' (148). Instead of simply describing the experience, Thoreau opts to describe his inner feelings about the exerience followed by an ambiguous statement about the unprecedended nature of this harmony. Through this use of vague language, Thoreau morphs a seemingly commonplace event into a novel intrigue. This string of causation relates to his purpose as it suggests that the mode of living in which Thoreau proposes leads to a life of everlasting fascination and awe. Thoreau himself confirms the truth of this relation when he writes that ``[his] life itself was become [his] amusement and never ceased to be novel. It was a drama of many scenes and without an end'' (63).

\newpage
\tq{14}{16 - The Pond in Winter}

% 14. Provide examples of an effective aspect of Thoreau’s style from two different passages in the text and analyze why it is effective, comparing and contrasting its use in the two passages.

One effective aspect of Thoreau's writing style is his use of italics on ordinary words to either emphasize surprising aspects of his descriptions or draw attention to his use of an important literary device. The use of italics for noteworthiness appears when Thoreau attempts to determine the shape of the bottom of Walden Pond: ``I laid a rule on the map lengthwise, and then breadthwise, and found, to my surprise, that the line of greatest length intersected the line of greatest breadth \emph{exactly} at the point of greatest depth, notwithstanding that . . . the outline of the pond far from regular, and the extreme length and breadth were got by measuring into the coves'' (156). This use of italics draws attention to the shock of Thoreau's discovery via an absolute description. While this example lacks any contribution to an argument, it succeeds in providing a more accurate depiction of Thoreau's mindset in addition to the event itself. Thoreau's use of italics as a highlighter for literary devices arises at the beginning of his chapter on Walden Pond in the winter when he awakes and describes a lingering dream: ``But there was dawning Nature, in whom all creatures live, looking in at my broad windows with serene and satisfied face, and no question on \emph{her} lips'' (153). The italics on the third-person singular genative pronoun call to attention Thoreau's personification of Nature as a woman who refuses to answer his questions. In contrast to the first example, this use of italics draws to attention a literary device instead of a surprising detail; however, both examples share a lack of contribution to a larger argument.

\newpage
\tq{2}{17 - Spring}

% 2. Analyze the relationship between exigence and kairos: how does Thoreau take advantage of the given situation during this time period and respond using the best available means of persuasion?

% ``In globe, \emph{glb}, the guttural \emph{g} adds tp the meaning the capcity of the throat. The feathers andwings of birds are still drier and thinner leaves. Thus, also, you pass from the lumpish grub in the earth to airy and fluttering butterfly. The very globe continually transcends and translates itself, and becomes winged in its orbit'' (165).

Published in 1854, \emph{Walden} takes advantage of the outrage of Northern citizens over the Compromise of 1850 and the Fugitive Slave Act by providing a method to escape from the chains of national interest. In order to address the various members of the populus to whom political propoganda appears ad nauseum, Thoreau portrays this retreat into nature as a less politcally charged method of civil disobeidence than that of his essay of the same name.\footnote{Thoreau, Henry. \emph{Resistance to Civil Government (Civil Disobedience)}, 1849.} As opposed to crafting his arguments using political ideology, in \emph{Walden} Thoreau uses metaphors strategically placed throughout his descriptions for the subtle effect of slowly convincing readers that his contention originated from within their own thoughts. An example of this kairotic rhetorical strategy arises when Thoreau describes the bank of Walden Pond during the spring and analyzes the sound of the word globe: ``In globe, \emph{glb}, the guttural \emph{g} adds to the meaning the capcity of the throat. . . . Thus, also, you pass from the lumpish grub in the earth to airy and fluttering butterfly. The very globe continually transcends and translates itself, and becomes winged in its orbit'' (165). Through this analysis, Thoreau echoes his main argument by allowing the interpretation of the ``guttural \emph{g}'' as the ideal mode of living in which he proposes. He further suggests this interpretation by directly addressing the reader as a ``lumpish grub in the earth'' or someone relying on the political climate for entertainment. In contrast, the imagery of an ``airy and fluttering butterfly'' elicits a feeling of freedom which suggests that this phrase represents the reader utilizing Thoreau's mode of living and transcending the reliance upon external, artificial forms of entertainment.

\newpage
\tq{15}{18 - Conclusion}

% 15. Note your impressions of the way the text ended, and analyze how this affects your reading of the text.

Thoreau's conclusion to \emph{Walden} summarizes the meat of all of his previous contentions in a method more succint than their original presentations. This shift into pure, uninterrupted argumentation stems from his completion of a description of his time at Walden Pond. However, the true genius of this final chapter appears when Thoreau gives the reader permission to live how he or she pleases when he writes, ``If you have built castles in the air, your work need not be lost; that is where they should be. Now put the foundations under them'' (174). This quotation demonstrates Thoreau's understanding that an effective argument need not be forced upon the arguee, as such diminishes the genuity of the contention.

What began in my mind as a perusal through a lifestyle of discomfort, slowly morphed into a message which resonated with me on the fundamental level of a fellow academic. After reading his suggestion of acting upon one's dreams, I was thoroughly convinced on the identification of Thoreau's \emph{Walden} as a classical work to be read for centuries to come. I cannot help but admire the authenticity in which Thoreau engrains in his writing, and the only word I have left as a description for this book is \emph{beautiful}.

\end{document}
