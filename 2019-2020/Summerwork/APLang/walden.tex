\documentclass[12pt]{article}

\usepackage[english]{babel}
\usepackage{indentfirst}

% - Margin - 1 inch on all sides
\usepackage[letterpaper]{geometry}
\usepackage[utf8]{inputenc}
\geometry{top=1.0in, bottom=1.0in, left=1.0in, right=1.0in}

% - Double Spacing -
\usepackage{setspace}
\doublespacing
\setstretch{2.00}

% FancyHDR settings
\setlength{\headheight}{15.2pt}
\usepackage{fancyhdr}
\pagestyle{fancy}
\fancyhf{}

% Indent First paragraph


% \tq{TQ num}{CH num}
\newcommand{\tq}[2]{
    \fancyhead[L]{James Harbour}
    \fancyhead[C]{\emph{Walden}: Reading Journals}
\fancyhead[R]{TQ:\ #1 $\vert$ CH: #2}
}

% Begin Document
\begin{document}
\raggedright\setlength{\parindent}{0.5in}


\tq{13}{1 - Economy}

After listing the various ways in which mankind suffers from uncertainty, Thoreau follows with a quotation from Confucius providing a mindset able to transcend this anxiety and affect positive change. Directly following a celebration of the miracle of contemplation, Thoreau writes, ``Confucius said, `To know that we know what we know, and that we do not know what we do not know, that is true knowledge'" (7). Thoreau utilizes the ethos of Confucius to emphasize the quintessential example of an individual with the ability to surpass the strain of everyday life. This ideal alleviates the pessimism of the negative musings in the preceding paragraph through the provision of a concrete mindset to strive towards. This paragraph, taken as a whole, formulates a method of resisting descent into nihilismtic pessimism whilst simultaneously building its ethos. Thoreau first presents his understanding of the reasons behind why mankind ``commit[s] [it]sel[f] to uncertainties'' (7) to later strengthen the credibility of his solution: that the act of contemplation directly combats nihilistic pessimism.

\newpage

\tq{1}{3 - Reading}


Through the use of first person narration, Thoreau crafts the narrative persona of an elder imparting knowledge to younger generations through both a description of his own life experiences and the creation of a feeling of collective responsibility with the use of the first person plural pronoun. An example of this dualistic impartment and acceptance of responsibility rises during Thoreau's admittance of his seldom reading of classical authors such as Plato: ``His Dialogues, which contain what was immortal in him, lie on the next shelf, and yet I never read them. . . . We should be as good as worthies of antiquity, but partly by first knowing how good they were'' (60).
% TODO Add about ethos by hope in mankind's ability.

Often when Thoreau crafts an argument, he proceeds to explain the counterargument in detail showcasing phronesis as he posesses the experience to understand not only his argument, but also the opposite of his argument. While






\end{document}
