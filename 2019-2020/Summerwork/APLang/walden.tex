\documentclass[12pt]{article}

%%%%%%%%%%%%%%%%%%%%%%%%%%%%%%%%%%-- Settings --%%%%%%%%%%%%%%%%%%%%%%%%%%%%%%%%%%%%%%%%%%%
\usepackage[english]{babel}

% - Margin - 1 inch on all sides
\usepackage[letterpaper]{geometry}
\usepackage[utf8]{inputenc}
\geometry{top=1.0in, bottom=1.0in, left=1.0in, right=1.0in}

% - Double Spacing -
\usepackage{setspace}
\doublespacing
\setstretch{2.00}

% FancyHDR settings
\setlength{\headheight}{15.2pt}
\usepackage{fancyhdr}
\pagestyle{fancy}
\fancyhf{}

% Indent First paragraph
\usepackage{indentfirst}

% Title Settings
\title{
    \vspace{2in}
    \textmd{\textbf{AP Language and Composition:\ Summerwork}}\\
    \normalsize\vspace{0.1in}\small{Due\ on\ August 12, 2019\ at 8:00pm}\\
    \vspace{3in}
}
\author{James Harbour}

% \tq{TQ num}{CH num}
\newcommand{\tq}[2]{
    \fancyhead[L]{\emph{Walden}: Reading Journals}
    \fancyhead[R]{TQ:\ #1 $\vert$ CH: #2}
}


%%%%%%%%%%%%%%%%%%%%%%%%%%%%%%%%%%-- Assignment --%%%%%%%%%%%%%%%%%%%%%%%%%%%%%%%%%%%%%%%%%%%

% http://www.gutenberg.org/files/205/205-h/205-h.htm

% 1. Describe the persona of the narrator (though Thoreau uses first person narration, he creates a narrative persona): what is revealed through tone, diction, syntax, etc. How does he create credibility (ethos: phronesis, arete and eunoia) and develop a relationship with the audience?
% 2. Analyze the relationship between exigence and kairos: how does Thoreau take advantage of the given situation during this time period and respond using the best available means of persuasion?
% 3. Who is the audience for the text (what are their beliefs, biases, background knowledge, etc.), and how does Thoreau adapt his message to this audience?
% 4. What rhetorical devices and strategies do you notice (refer to the rhetorical terms which follow), and how do these contribute to Thoreau’s argument and purpose?
% 5. How do aspects of narrative development contribute to Thoreau’s argument and purpose?
% 6. Analyze the text’s argument: Is it implicit or explicit? How does Thoreau back his claim?
% 7. Analyze Thoreau’s purpose: is it a question of fact, definition, quality or policy (stasis theory)? How do the rhetorical choices Thoreau makes relate to his purpose?
% 8. Conduct a detailed analysis of a short passage in terms of syntax. Describe the syntax and relate to Thoreau’s purpose.
% 9. Analyze a time when your reading changed (for example, you recognize a pattern, the text suddenly seems to be about something different from what you originally thought, you discover that you were misreading, you realize Thoreau has introduced a new context or perspective, or you were surprised or puzzled). Describe your thought process and realizations.
% 10. Analyze Thoreau’s philosophy of transcendentalism using textual evidence.
% 11. Analyze details that seem important and that make you take a second look: how does this relate to Thoreau’s purpose?
% 12. Analyze how Thoreau uses ambiguity and relate to his purpose.
% 13. Select a passage in which either dialogue (see the chapter “Brute Neighbors”) or quotations are used for a particular effect and analyze the effect.
% 14. Provide examples of an effective aspect of Thoreau’s style from two different passages in the text and analyze why it is effective, comparing and contrasting its use in the two passages.
% 15. Note your impressions of the way the text ended, and analyze how this affects your reading of the text.


%%%%%%%%%%%%%%%%%%%%%%%%%%%%%%%%%%-- Document --%%%%%%%%%%%%%%%%%%%%%%%%%%%%%%%%%%%%%%%%%%%

% Begin Document
\begin{document}
\maketitle
\pagebreak
\raggedright\setlength{\parindent}{0.5in}


\tq{13}{1 - Economy}
% 13. Select a passage in which either dialogue (see the chapter “Brute Neighbors”) or quotations are used for a particular effect and analyze the effect.

After listing the various ways in which mankind suffers from uncertainty, Thoreau follows with a quotation from Confucius providing a mindset able to transcend this anxiety and affect positive change. Directly following a celebration of the miracle of contemplation, Thoreau writes, ``Confucius said, `To know that we know what we know, and that we do not know what we do not know, that is true knowledge'" (7). Thoreau utilizes the ethos of Confucius to emphasize the quintessential example of an individual with the ability to surpass the strain of everyday life. This ideal alleviates the pessimism of the negative musings in the preceding paragraph through the provision of a concrete mindset to strive towards. This paragraph, taken as a whole, formulates a method of resisting descent into nihilismtic pessimism whilst simultaneously building its ethos. Thoreau first presents his understanding of the reasons behind why mankind ``commit[s] [it]sel[f] to uncertainties'' (7) to later strengthen the credibility of his solution: that the act of contemplation directly combats nihilistic pessimism.


\newpage
\tq{TBD}{2 - Where I Lived, and What I Lived For}

% 10. Analyze Thoreau’s philosophy of transcendentalism using textual evidence.

% Through his explanation of his residence in nature, Thoreau creates a perfect setting for the independent aquisition of the philosophy of transcendentalism.

\LaTeX\ placeholder

\newpage
\tq{1}{3 - Reading}

% 1. Describe the persona of the narrator (though Thoreau uses first person narration, he creates a narrative persona): what is revealed through tone, diction, syntax, etc. How does he create credibility (ethos: phronesis, arete and eunoia) and develop a relationship with the audience?

Through the use of first person narration, Thoreau crafts the narrative persona of an elder imparting knowledge to younger generations through both a description of his own life experiences and the creation of a feeling of collective responsibility with the use of the first person plural pronoun. An example of this dualistic impartment and acceptance of responsibility rises during Thoreau's admittance of his seldom reading of classical authors such as Plato: ``His Dialogues, which contain what was immortal in him, lie on the next shelf, and yet I never read them. . . . We should be as good as worthies of antiquity, but partly by first knowing how good they were. We are a race of titmen, and soar but little higher in our intellectual flights than the columns of the daily paper'' (60). By creating a feeling of responsibility alongside the rest of mankind, Thoreau morphs this otherwise negative statement into an insight suggesting hope for the future. In this way, he creates ethos through eunoia; however, Thoreau also relies on other methods of forging his credibility. Often when Thoreau crafts an argument, he proceeds to explain the counterargument in detail showcasing phronesis as he posesses the experience to understand not only his argument, but also the opposite of his argument.

\newpage
\tq{8}{4 - Sounds}

% 8. Conduct a detailed analysis of a short passage in terms of syntax. Describe the syntax and relate to Thoreau’s purpose.3

% TODO add intro to quotation
 Out of the many passages in \emph{Walden} lending themselves to syntactical analysis, Thoreau's most poigniant thus far appears near the beginning of his chapter on sounds: ``I had this advantage, at least, in my mode of life, over those who were obliged to look abroad for amusement, to society and the theatre, that my life itself was become my amusement and never ceased to be novel. It was a drama of many scenes and without an end'' (63). Thoreau's use of irregular syntax in this passage cements the stark contrast between the perpetual novelty of his everyday life and the fleeting amusement found by a society that seeks entertainment from the external. The first example of this peculiar prose arises in the syntactical choice to use ``was become'' instead of the present perfect ``has become.'' In doing so, Thoreau draws attention to the anteriority of his life becoming an amusement, thus increasing the magnitude of the period of novelty he finds throughout his time at Walden Pond and comparatively morphing the effect of external entertainment into that of infinitesmal significance. One other notable syntactial choice appears in Thoreau's literal separation of the theatre from society. Although a product of society, theatre represents society's attempt to escape the tedium of societal obligation; moreover, Thoreau sees this attempt as having failed due to its brevity and uses the subject as a metaphor to liken his life in nature to that of a drama which succeeds in creating perpetual entertainment.


\end{document}
