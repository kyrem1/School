\documentclass{scrartcl}
\usepackage[sexy]{james}
\title{
    \vspace{2in}
    \textmd{\textbf{MGF 3301 :\ Bridge to Abstract Mathematics}}\\
    \vspace{0.1in}\large{Professor: \textit{Brendan Nagle}}\\
    \vspace{3in}
    Notes by James Harbour
}
\begin{document}
\maketitle
\newpage

\section{Exam 1 Proofs}

\begin{theorem}[Bezout's Identity]
  For all $a, b\in \N$,
  \[
    \gcd{(a,b)} = \min\{as+bt : s,t\in\Z\ \mathrm{satisfy}\ as+bt>0\}
  \]
\end{theorem}

\begin{proof}
  Let $a,b \in \N$ be positive integers and let $A = \{as+bt : s,t\in\Z\ \mathrm{satisfy}\ as+bt>0\}$. Set $D = \gcd{(a,b)}$. Because $A \subseteq \N$ and trivially $A \neq \nullset$, by the Well Ordering Principle there is a smallest element in $A$. Thus, set $m = \min{(A)}$. We now continue by proving two claims.

  \begin{claim}[Easy Part]
    $D \leq m$
  \end{claim}

  \begin{subproof}
    Let $s_0, t_0 \in \Z$ satisfy $m = as_0 + bt_0$. Since $D \mid a$ and $D \mid b$, it follows that $D \mid (as_0 + bt_0)$. Thus, by our hypothesis, $D \mid m$, so by definition $D \leq m$.
  \end{subproof}

  \begin{claim}[Hard Part]
    $D \geq m$
  \end{claim}

  \begin{subproof}
    Consider that $(m\mid a \land m\mid b) \implies m\mid D \implies m \leq D$. So we show that $m\mid a \land m\mid b$. Assume, for the sake of contradiction, that \WLOG, $m \nmid a$.\\
    By division with remainder, $\exists\ q=q_{m,a} \in \Z$ and $\exists\ r=r_{m,a} \in \Z$ such that
    \begin{equation}\label{eq:1}
      a = qm + r,\qquad 0 \leq r < m
    \end{equation}
    In fact, because $m\nmid a$, we know that $0 \leq r < m$. Since $m=as_0 + bt_0$,
    \begin{align*}
      a &= qm + r \\
      &\by{\eqref{eq:1}}{=} q(as_0 + bt_0) + r \\
      &=(qs_0)a + (qt_0)b + r \\
      r &= (1 - qs_0)a + (-qt_0)b
    \end{align*}
    So, $r$ is an integer linear combination of $a$ and $b$ because $1 - qs_0 \in \Z$ and $-qt_0 \in \Z$. By division with remainder, $r > 0$ and $r < m$. So it follows that $r \in A$, however by the well ordering principle, m is the smallest element of $A$, thus we have reached a contradiction.
  \end{subproof}
\end{proof}

\begin{lemma}[Euclid's Lemma]
  For all $a, b, p \in \N$, if $p$ is prime and $p \mid ab$, then $p \mid a$ or $p \mid b$.
\end{lemma}

\begin{proof}
  Let $a,b \in \N$ and let $p\in\N$ be prime with $p \mid ab$. If $p \mid a$ or $p \mid b$, then we are done. So, assume \WLOG\ that $p \nmid a$. Because $p$ is prime, its only divisors are $p$ and $1$, and since $p \nmid a$, it follows that $\gcd{(p, a)} = 1$. Thus, by Bezout's Identity, $\exists\ s,p \in \Z$ such that
  \begin{align*}
    as + pt &= 1 \\
    abs + pbt &= b
  \end{align*}
  Because $p \mid ab\implies p \mid abs$ and $p \mid p \implies p \mid p(bt)$, we have that $p \mid (abs + pbt)$. Thus $p \mid b$.
\end{proof}
\newpage
\section{Exam 2 Proofs}

\begin{theorem}[Division with Remainder]
  For all integers $a,b \in \Z$, where $a \neq 0$, there exist unique integers $q,r \in \Z$ such that \[ b = aq+r, \quad \mathrm{where\ } 0\leq r < |a|. \]
\end{theorem}

\begin{proof}[Existence Portion]
  Let integers $a,b \in \Z$ be given, where $a \neq 0$. If $a \mid b$, then there exists some integer $k \in \Z$ such that $b = ka$. By choosing $q=k$ and $r=0$, we have that $b =aq+r$ with $0\leq r=0 < |a| \neq 0$ and are done. Thus, suppose that $a \nmid b$, with $a \neq 1$ and $b \neq 1$.

  Consider the sets
  \[\cR = \left\{b - am : m \in \Z \right\} \quad \mathrm{and} \quad \cR^{+} = \cR \cap \N.\]

  \begin{claim}[W.O.P. Condition]
    $\cR^{+} \neq \emptyset$
  \end{claim}

  \begin{subproof}
    Choose $m = m_0 = -\displaystyle\frac{|a||b|}{a}$. Because $\displaystyle\frac{|a|}{a}=\pm 1 \in \Z$, we have that $m_0 = \pm |b| \in \Z$; thus $b-am_0 \in \cR$. With this choice, we have that
    \begin{align*}
      b - am_0 &= b - a\left(\frac{-|a||b|}{a}\right) \\
      &= b + |a||b|  \\
      &\geq -|b| + |a||b|  \\
      &=|b|\left(|a| - 1\right) \by{\mathrm{hyp}}{\geq} 1
    \end{align*}
    Thus $b-am_0 \in \cR^{+}$, so $\cR^{+} \neq \emptyset$.
  \end{subproof}

  Because $\emptyset \neq \cR^{+} \subseteq \N$, the Well-Ordering Principle guarantees that $\cR^{+}$ admits a least element $r_0=\min{\cR^+}$, where we write $r_0 = b - aq_0$ for some positive integer $m = q_0 \in \Z$. Thus $b=aq_0+r_0$ with $r_0 \geq 1 > 0$ because $r_0 \in \cR$. We must now show that $r_0 < |a|$.

  \begin{claim*}
    $r_0 < |a|$
  \end{claim*}
  \begin{subproof}
    Suppose, for the sake of contradiction, that $r_0 \geq |a|$. If $r_0 = |a|$, then
    \begin{align*}
      b &= aq_0 + r_0 \\
      &= aq_0 + |a| \\
      &= a\left(q_0 + \frac{|a|}{a}\right) = a\left(q_0 \pm 1\right)
    \end{align*}
    Thus $a \mid b$, which is a contradiction. If $r_0 > |a|$, i.e. $r_0 \geq |a| + 1$, then
    \begin{align*}
      b &= aq_0 + r_0 \\
      & \geq aq_0 + |a| + 1 \\
      &= a\left(q_0 + \frac{|a|}{a}\right) + 1
    \end{align*}
    And so the integer
    \begin{equation*}
      b - a\left(q_0 + \frac{|a|}{a}\right) \geq 1
    \end{equation*}
    belongs to $\cR^{+}$. On the other hand,
    \begin{align*}
      b - a\left(q_0 + \frac{|a|}{a}\right) &= b - aq_0 - |a| \\
      &< b - aq_0 = r_0
    \end{align*}
    which is a contradiction, as $r_0$ is the smallest element of $\cR^{+}$. Thus, $r_0 < |a|$
  \end{subproof}
\end{proof}

\end{document}
