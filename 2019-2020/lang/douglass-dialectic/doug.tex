\documentclass[12pt]{article}

\usepackage[english]{babel}
\usepackage{indentfirst}

% - Margin - 1 inch on all sides
\usepackage[letterpaper]{geometry}
\usepackage[utf8]{inputenc}
\geometry{top=1.0in, bottom=1.0in, left=1.0in, right=1.0in}
\usepackage{longtable}

% FancyHDR settings
\setlength{\headheight}{15.2pt}
\usepackage{fancyhdr}



% Begin Document
\begin{document}
\begin{titlepage}
    \begin{center}
        \vspace*{1cm}
        \Large \emph{NLFD} (54-70): Dialectic Journal
        \vfill
        \large
        James Harbour\\
        \vspace{0.4cm}
        AP: Language and Composition\\
        \vspace{0.4cm}
        Dr. Van Den Broecke\\
        \vspace{0.4cm}
        November 3, 2019 \\
        \vspace{1.6cm}
        \normalsize
        Document typeset with TeX-Live and compiled with pdflatex.
        \vspace{0.5cm}
    \end{center}
\end{titlepage}

% Begin Blank Page
\pagestyle{empty}
\begin{center}
  \vspace{1cm}
  This page has intentionally been left blank.
  \vfill
\end{center}

\newpage
% End Blank Page

% Begin Main Document
\pagestyle{fancy}
\fancyhf{}
\fancyhead[L]{James Harbour}
\fancyhead[C]{\emph{NLFD} (54-70): Dialectic Journal}
\fancyhead[R]{(\thepage)}

\begin{longtable}{| p{.40\textwidth} | p{.55\textwidth} |}
  \hline
    \begin{center}
      ``I was made acquanted with his wife not less than with himself. They were well matched, being equally mean and cruel'' (54).
    \end{center} & This description provides a juxtaposition between the cruelty of the wife of his current master, and that of his master in Baltimore. Douglass includes this description to further emphasize the drastic regional differences of the treatment of slaves.   \\
  \hline
    \begin{center}
      ``He was a slaveholder without the ability to hold slaves'' (55).
    \end{center} & Utilizing chiasmus, Douglass perfectly describes the disposition of Thomas Auld as a slaveholder. Consequentially, Douglass summarizes the inherently contradictory nature of adopted slaveholders as a whole with an equally contradictory remark on their ability to hold slaves.    \\
  \hline
    \begin{center}
      ``Prior to his conversion, he relied upon his own depravity to shield and sustain him in his savage barbarity; but after his conversion, he found religious sanction and support for his slaveholding cruelty.'' (56)
    \end{center} & Douglass gives a prelude to his appendix regarding his disposition towards Christianity. In this quotatiuon, Douglass uses parallel structure to  contrast Auld's rationales pre and post-conversion whilst simultaneously presenting the similarity of these rationales' resulting actions.  \\
  \hline
    \begin{center}
      ``I was aware of all the facts, having been made acquanted with them by a young man who had lived there. I nevertheless made the change gladly; for I was sure of getting enough to eat'' (58).
    \end{center} & By preempting this statement with a detailed description of the horrors of Mr. Covey, Douglass explicates magnitude of his hunger under Thomas Auld.   \\
  \hline
    \begin{center}
      ``My natural elasticity was crushed, my intellect languished, the disposition to read departed, the cheerful spark that lingered about my eye died; the dark night of slavery closed in upon me; and behold a man transformed into a brute!'' (63)
      \end{center} & With this polysyndeton of various metaphors, Douglass paints a vivid picture of his descent into the depths to provide ample contrast to the intelligible form he is often viewed as.   \\
  \hline
    \begin{center}
      ``Thus I used to think, and thus I used to speak to myself; goaded almost to madness at one moment, and at the next reconciling myself to my wretched lot'' (64).
    \end{center} & This commentary on his previous monologue presents the beginnings of Douglasses inner confliction between escape and acceptance. He presents the two arguments in a concise, yet profound manner and displays through explanation the outward effects of this inner struggle. \\
  \hline
    \begin{center}
      ``You have seen how a man was made a slave; you shall see how a slave was made a man'' (64).
    \end{center} & This poigniant declaration to the reader provides a heading for the events under Mr. Covey whilst simultaneously using parallel structure to prompt Douglass's both metaphorical and literal escape from slavery.  \\
  \hline
    \begin{center}
      ``To please him, I at length took the root, and, according to his direction, carried it upon my right side'' (68).
    \end{center} & With this inclusion, Douglass intends a rhetorical effect of creating a similarity between him and his audience who he believes could relate to succumbing to superstition from time to time.\\
  \hline
    \begin{center}
      ``This battle with Mr. Covey was the turning-point in my career as a slave'' (69).
    \end{center} & Worried the reader might miss this profound moment, Douglass includes an explicit suggestion of a transition for his narrative and thus allows easier reading and chronological sorting. \\
  \hline
    \begin{center}
      ``But by far the larger part engaged in such sports and merriments as playing ball, wrestling, running footraces, fiddling, dancing, and drinking whiskey; and this latter mode of spending the time was by far the most agreeable to the feelings of our masters'' (70).
    \end{center} & Douglass includes this depiction of time spend during holidays to further humanize the acts of slaves to his audience. By presenting an image of their similarities, Douglass creates a rhetorical opening by which he uses to inject his argument.  \\
  \hline
\end{longtable}






\end{document}
